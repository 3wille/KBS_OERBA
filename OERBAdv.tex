\documentclass{beamer}
\setbeamertemplate{navigation symbols}{}

\usepackage[normalem]{ulem}
\usetheme{Boadilla}
\usepackage{ngerman}
\usepackage{algorithmicx}
\usepackage[utf8]{inputenc}
\usepackage[T1]{fontenc}
\usepackage{sansmathaccent}
\pdfmapfile{+sansmathaccent.map}
%\usepackage{listings}
%\usepackage{minted}
\beamersetuncovermixins{\opaqueness<1>{25}}{\opaqueness<2->{15}}
\begin{document}
    \title{OERB Advanced}
    \author{Frederik Wille}

    \date{\today}

    \begin{frame}
        \titlepage
        {\tiny nach Folien von Dennis 6keitzel}
    \end{frame}

    \begin{frame}
        \frametitle{Inhalt}
        \tableofcontents
    \end{frame}

    \section{Grundlagen}
    \subsection{Linux}
    \begin{frame}
        \frametitle{Dateisystem und Pathing}
        \begin{block}{Root}
            $/$
        \end{block}
        \begin{block}{Ordner}
            uni/KBS
        \end{block}
        \begin{block}{Home}
            /home/freddy \\
            $\tilde \ $
        \end{block}
    \end{frame}

    \begin{frame}
        \frametitle{Rechte}
        \begin{itemize}
            \item user and group
            \item read / write / execute
        \end{itemize}
        \begin{block}{Beispiel}
            drwxr-xr-x\ \ 3\ kbs\ \ \ \ kbs\ \ \ \ 4.0K Oct 17 17:38 kbs/
        \end{block}
    \end{frame}

    \begin{frame}
        \frametitle{spezielle Dateien/Ordner}
        \begin{block}{aktueller Ordner}
            $.$
        \end{block}
        \begin{block}{übergeordneter Ordner}
            $..$
        \end{block}
        \begin{block}{unsichtbare Dateien/Ordner}
            $.ssh$
        \end{block}
    \end{frame}

    \subsection{Shell}
    \begin{frame}
        \frametitle{Navigation}
        \begin{block}{Completion}
            \begin{Huge}
                $\rightleftarrows$TAB
            \end{Huge}
        \end{block}
        \begin{block}{aktueller Pfad}
            \$ pwd
        \end{block}
        \begin{block}{Change Directory}
            \$ cd [directory]
        \end{block}
        \begin{block}{List}
            \$ ls
        \end{block}
    \end{frame}

    \begin{frame}
        \frametitle{Dateien/Ordner}
        \begin{block}{Ordner erstellen/l\"oschen}
            \$ mkdir [name] \\
            \$ rmdir [name]
        \end{block}
        \begin{block}{Datei erstellen/l\"oschen}
            \$ touch [name] \\
            \$ rm [name]
        \end{block}
    \end{frame}

    \begin{frame}
        \frametitle{Weitere Programme}
        \begin{itemize}
            \item man
            \item Editoren
            \begin{itemize}
                \item vim
                \item gedit
                \item nano
            \end{itemize}
            \item less
            \item sudo
            \item (h)top
            \item ps (aux)
            \item grep
            \item wget
            \item kill(all)/pkill
        \end{itemize}
    \end{frame}

    \section{Zugang ins IRZ Netz}
    \begin{frame}
        \frametitle{Zugang ins IRZ Netz}
        \begin{itemize}
            \item VPN
            \item SSH
            \item bestimmte RJ45 Dosen?
        \end{itemize}
    \end{frame}

    \subsection{VPN}
    \begin{frame}
        \frametitle{Was?}
        \begin{itemize}
            \item System ,,sieht'' anderes Netz ($\rightarrow$ andere IP-Adresse)
            \item Zugriff auf nicht öffentliche Dienste
        \end{itemize}
    \end{frame}
    \begin{frame}
        \frametitle{Warum?}
        \begin{itemize}
            \item Drucken (vor allem Windows)
            \item interne Seiten
        \end{itemize}
    \end{frame}
    \begin{frame}
        \frametitle{Wie?}
        \begin{itemize}
            \item GUI $\longrightarrow$ Wiki
            \item CLI $\longrightarrow$ Wiki (und Menschen)
        \end{itemize}
    \end{frame}
    \begin{frame}
        \frametitle{Bemerkungen}
        \begin{itemize}
            \item Internet-Zugang kann langsamer werden
            \item kann überlasten
            \item bestimmte Ports blockiert
        \end{itemize}
    \end{frame}
    \subsection{SSH}
    \begin{frame}
        \frametitle{Was?}
        \begin{itemize}
            \item \textbf{S}ecure \textbf{Sh}ell
            \item (Fern)zugriff auf unix-ähnliche Systeme
            \item Installation
            \begin{itemize}
            \item Unix: done
            \item Windows: putty oder cygwin
            \end{itemize}
        \end{itemize}
    \end{frame}
    \begin{frame}
        \frametitle{Wie?}
        \begin{itemize}
            \item Host: rzssh1.informatik.uni-hamburg.de
            %\item Fingerprint: ECDSA: SHA256:mMfqLOIIj4spGkmKC3gvYpXqRI3/K8waCYGBqXGfZGk
            \item User\&Password: IRZ-Kennung (5musterm)
            \begin{block}{ECDSA Fingerprint}
                SHA256:mMfqLOIIj4spGkmKC3gvYpXqRI3/K8waCYGBqXGfZGk
            \end{block}
        \end{itemize}
    \end{frame}
    \begin{frame}
        \frametitle{Key}
        \begin{block}{Shell}
            \$ ssh-keygen
        \end{block}
        \pause
        \begin{block}{Enter}
            Generating public/private rsa key pair. \\
            Enter file in which to save the key (/home/kbs/.ssh/id\_rsa):
        \end{block}
        \pause
        \begin{block}{Passwort}
            Created directory \'/home/kbs/.ssh\'. \\
            Enter passphrase (empty for no passphrase):\\
            Enter same passphrase again: \\
        \end{block}
        %\begin{algorithmic}
        %    \State ssh-keygen
        %\end{algorithmic}
        %\begin{verbatim}
        %\end{verbatim}
        %\begin{verbatim}
        % \begin{lstlisting}[frame=single]
        %     ssh-keygen
        % \end{lstlisting}
        % \begin{lstlisting}
        % ssh-copy-id -i ~/.ssh/id_rsa.pub 5musterm@rzssh1.informatik.uni-hamburg.de
        % \end{lstlisting}
        %\end{verbatim}
    \end{frame}
    \begin{frame}
        \frametitle{Key}
        \begin{block}{Fingerprint}
            The key fingerprint is: \\
            SHA256:3InyN8rMGpoyV6KGRBV8cH0gzCmVoXz3GHnNMJpQoUE kbs@rickenbacker\\
            The key's randomart image is:\\
            \begin{verbatim}

            +---[RSA 2048]----+\\
            |\ \ .oBE*+oo\ \ \ \ \ \ \ |\\
            |\ \ .+o*=.+.=\ \ \ \ \ \ |\\
            |\ \ .ooo\ *..\ o\ \ \ \ \ |\\
            |\ .\ \ .\ ..=o\ .\ \ \ \ \ |\\
            |.\ \ \ \ \ ..S.o\ \ \ \ \ \ |\\
            |\ .\ \ .\ .o\ \ \ \ \ \ \ \ \ |\\
            |..\ .\ o\.\ .\ o\ \ \ \ \ \ |\\
            |.\ =\ .o\ =\ o\ .\ \ \ \ \ |\\
            |\ .\ +o\ ..=\ \ \ \ \ \ \ \ |\\
            +----[SHA256]-----+
        \end{verbatim}
        \end{block}
    \end{frame}
    \begin{frame}
        \frametitle{Copy ID}
        \begin{block}{shell}
            \small{\$ ssh-copy-id -i ~/.ssh/id\_rsa.pub 5musterm@rzssh1.informatik.uni-hamburg.de}

        \end{block}
    \end{frame}
    \begin{frame}
        \frametitle{Userconfig}
        \begin{block}{$\tilde \ $/.ssh/config}
            Host fbi\\
            \qquad Hostname rzssh1.informatik.uni-hamburg.de\\
            \qquad User 5musterm\\
            \qquad \# ForwardX11 yes\\
            \qquad DynamicForward 7777\\
        \end{block}
    \end{frame}
\section{Git}
\begin{frame}
    \frametitle{Was?}
    \begin{itemize}
        \item asdf
    \end{itemize}
\end{frame}

\end{document}
